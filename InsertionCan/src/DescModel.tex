
\newcommand\ModelDescription[0]{ 
A node to insert is a place with a token.\\
The model of a node insertion in a CAN network is divided into an insertion in an empty CAN and an insertion in a non empty CAN. To do so, we use a counter place and a place with a token to check the CAN network status.\\
The insertion in an empty CAN can be achieved without any constraint. Once the insertion done, the counter place takes one more token.\\
On the other hand, the insertion in a non empty CAN leads to an "AskSharing" place. At this point, the node has to ask an other node to share its space in the CAN. A node can send its request to one inserted node only.\\
If N is the number of nodes to insert, a node can choose one inserted node among all other nodes. This is why we have N-1 possible paths to ask a request. And for the same reason, we also have N-1 paths for entering answer of the request.\\
We have to note that this model uses a multi-stride and sequential functioning on this part of the process.\\
The answer may be an acceptation -so the node will be inserted- or a denial -the node can retry its insertion by repeating the process. \\

H: First insertion is always in an empty CAN\\
H: Processes exchange via asynchronous communication\\

}

\newcommand\ModelReference[0]{
This model was described in:
Sylvia Ratnasamy,Paul Francis,Mark Handley,Richard Karp,Scott Shenker (2001). {\em A Scalable Content-Addressable Network}. \\
\url{https://dl.acm.org/doi/pdf/10.1145/383059.383072}
}


