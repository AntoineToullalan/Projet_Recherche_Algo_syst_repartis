\documentclass[french]{beamer}

\usepackage[utf8]{inputenc}
\usepackage[T1]{fontenc}
\usepackage{lmodern}
\usepackage{amsmath, amssymb}

\usepackage{babel}


%CHOIX DU THEME et/ou DE SA COULEUR
% => essayer différents thèmes (en décommantant une des trois lignes suivantes)
\usetheme{PaloAlto}
%\usetheme{Madrid}
%\usetheme{Copenhagen}

% => il est possible, pour un thème donné, de modifier seulement la couleur
%\usecolortheme{crane}
%\usecolortheme{seahorse}

%\useoutertheme[left]{sidebar}


%Pour le TITLEPAGE
\title{Projet de Recherche}
\author[]{TOULLALAN Antoine MENDAS Rosa}
\date{Lundi 15 Février (4e réunion)}


\begin{document}

\begin{frame}
	\titlepage
\end{frame}

\begin{frame}{Réseau de Pastry}
	\tableofcontents
\end{frame}
\section{Description d'un noeud}
\begin{frame}{Description d'un noeud}
Chaque noeud dans le réseau contient 3 structures de données qui constituent son \textbf{état}: Un \textbf{Leaf Set}, Un ensemble de \textbf{voisins} et une \textbf{table de routage}.\\
Chacune de ces structures contenant l'adresse IP et le NodeID d'autres noeuds du réseaux (cela ne constitue qu'un petite partie de l'ensemble des noeuds du réseau).\\Ainsi chaque noeud peut communiquer avec d'autres noeuds et ainsi envoyer et recevoir des messages.
\end{frame}
\begin{frame}{L'état d'un noeud}
\framesubtitle{Leaf Set}<1>
\framesubtitle{Voisins}<2>
\framesubtitle{Table de Routage}<3>
	On prend l'exemple d'un noeud avec le NodeID: \textbf{10233102}
	\includegraphics[scale=0.35]{"Leaf_set"}<1>
	\visible<1>{La taille du Leaf Set est 8, on a donc 8/2=4 NodeID inferieurs à 10233102 et  8/2=4 NodeID supérieurs à 10233102}
	\includegraphics[scale=0.35]{"voisins"}<2>
	\includegraphics[scale=0.35]{"table_routage"}<3>
\end{frame}

\section{Routage dans le réseau}
\begin{frame}{Comment utiliser la table de routage ?}
\framesubtitle{Un algorithme Greedy}<2>
	\visible<1,2>{-On a vu que chaque noeud possède une table de routage de 32 lignes et 15 colonnes soit 480 case, chacune contenant l'adresse IP et le NodeID d'un noeud.\\Mais comment ce noeud va utiliser cette structure pour envoyer et recevoir des messages ?\\}

	\visible<2>{-\textbf{Un algorithme Greedy} est utilisé par chaque Noeud qui reçoit une requête à destination d'un NodeID,\\ la requête peut aussi être associée à l'ObjID d'une donnée, dans ce cas la destination est le NodeID le plus proche de l'ObjID dans le réseau, cette donnée sera copiée par k noeuds de son Leaf Set.}
\end{frame}

\begin{frame}{Comment utiliser la table de routage ?}
\framesubtitle{Un algorithme Greedy}
	\visible<1,2>{-Soit la requête à destination du NodeID D qui arrive dans le noeud de Node ID A avec A!=D\\}
	\visible<2>{-\textbf{ 1ere etape : }A regarde si D est compris entre le plus petit NodeID du Leaf Set et le plus grand nodeID, si oui la requête est transmise à l'adresse dans le LeafSet dont le NodeID est le plus proche de D\\}
	\includegraphics[scale=0.35]{"Leaf_set"}<2>
\end{frame}
\begin{frame}{Comment utiliser la table de routage ?}
	-\textbf{ 2eme etape : } Sinon on utilise la table de routage\\Soit l la longueur du préfixe en commun entre A et D et Dl la valeur du symbole num l dans D\\On a $ R_l^Dl $ la case de ligne l et colonne Dl dans la table de routage (on indexe les lignes et colonnes à partir de 0). si la case contient un NodeID on transmet la requête vers le noeud correpondant. 
\end{frame}
\begin{frame}{Comment utiliser la table de routage ?}
\framesubtitle{Exemple}
	\includegraphics[scale=0.35]{"table_routage"}
ex: Si on a A=\textbf{1023}3102 et D=\textbf{1023}2423 alors le préfixe en commun est 1023 de longueur 4 donc l=4. La valeur num 4 de D est 2 donc  Dl=2.Donc $ R_l^Dl $ est $ R^2_4 $, on transmet donc la requête au noeud de NodeID \textbf{10232}121 On s'est rapproché d'un symbole vers le NodeId recherché.
\end{frame}
\begin{frame}{Comment utiliser la table de routage ?}
\framesubtitle{Exemple}
\visible<1>{-\textbf{ 3eme etape (plutôt rare) : } si la case $ R_l^Dl $ est vide, on choisit un autre NodeId T dans la table de routage, Leaf Set ou voisins tel que le préfixe en commun entre T et D ait une longueur supérieure à l.\\}
\visible<2>{\textbf{Conclusion :\\} Si la case $ R_l^Dl $ n'est pas vide (ce qui arrive la plupart du temps)  avec T le nodeID dans la case.La requête se rapproche alors d'un symbole vers le NodeId recherché. Donc la table de routage permet bien à la requête de se rapprocher de l'objectif.}
\end{frame}

\section{Perte d'un noeud}
\begin{frame}{Perte d'un noeud}
\begin{large}
On considère qu'il y a perte d'un noeud A lorsque les noeuds ne reçoivent pas de réponse lorsqu'ils communiquent avec ce noeud.\\ Dans ce cas il faut mettre à jour les états contenant A parmi les noeuds du réseau. Il faut donc mettre à jour le \textbf{Leaf Set}, la \textbf{table de routage} et les \textbf{voisins} pour ces noeuds.
\end{large}
\end{frame}

\begin{frame}{Perte d'un noeud}
\framesubtitle{Mise à jour du Leaf Set}
Pour mettre à jour le Leaf Set de B qui contient A, B va choisir dans son Leaf Set le noeud C qui a le nodeID le plus grand dans le côté ou se situe B. \\
Par ex: \\si A=Li avec $ \dfrac{-|L|}{2} $<i<0 alors on choisit C=$ L_{-|L|/2} $\\
si A=Li avec 0<i<$ \dfrac{|L|}{2} $ alors on choisit C=$ L_{+|L|/2} $\\
Après qu'on a choisi le noeud C, B lui demande son Leaf Set qu'on nomme L'. B choisit des noeuds de L' à ajouter à son Leaf Set L après avoir vérifié qu'ils n'étaient pas perdu en les contactant (dans la limite de la taille du Leaf Set).
\end{frame}
\begin{frame}{Perte d'un noeud}
\framesubtitle{Mise à jour de la table de routage}
\visible<1,2>{Pour remplacer A dans la table de routage de B, si A est contenu dans la case $ R_l^Dl $ de la table de B, alors B va contacter un autre noeud sur la même ligne (donc la ligne l) , on appelle ce noeud C. B demande à C le contenu de sa case $ R_l^Dl $ qui devient la nouvelle case $ R_l^Dl $ de B. }\visible<2>{\textit{\\Ainsi la nouvelle case $ R_l^Dl $ dans la table de B conserve le même préfixe en commun avec le NodeID de B.}}
\end{frame}

\begin{frame}{Perte d'un noeud}
\framesubtitle{Mise à jour de l'ensemble des voisins}
Si A est un voisin de B, B demande à ses autres voisins leur propres voisins, et B ajoute dans ses voisins les noeuds qui lui sont envoyés et qui sont les plus proches de lui (dans la limite de la taille de sa table de voisins).
\end{frame}

\section{Ajout d'un noeud}
\begin{frame}{Ajout d'un noeud}
\begin{small}
Supposons que le noeud X veuille joindre un réseau de Pastry. L'état de X contient une table de routage, un leaf set et un ensemble de voisins vierge.\\

\visible<2,3,4,5,6,7>{\textbf{1ere étape :} Une clé de 128 bits est associée à X\\}
\visible<3,4,5,6,7>{\textbf{2eme étape :} X localise un noeud A proche géographiquement.\\A diffuse la requête de X au noeud Z , le noeud Z étant le noeud plus proche de X parmi les voisins A au niveau de la clé\\ }
\visible<4,5,6,7>{\textbf{3eme étape :} A,Z et les noeuds constituant le chemin entre A et Z envoient leur table de routage à X-> X initialise sa table de routage.\\} 
\visible<5,6,7>{\textbf{4eme étape :} X informe tout les noeuds dans sa nouvelle table de routage de son arrivée\\}
\visible<6,7>{\textbf{5eme étape :} comme A est le noeud le plus proche géographiquement  de X, A transmet sa table de voisin à X. Et Comme Z est le noeud le plus proche au niveau de la clé, il envoie à X son Leaf Set.\\}
\visible<7>{\textbf{dernière étape :} X transmet une copie de sa table de routage à tout les noeuds de son Leaf Set, sa table de routage et  l'ensemble de ses voisins. Ces noeuds utilisent cette table pour mettre à jour leur état.\\}
\end{small}
\end{frame}
\section{Conciliation des copies}
\begin{frame}{Conciliation des copies}
\framesubtitle{Replication des données}
On a déjà vu que lorsqu'une donnée est stockée dans le réseau elle est répliquée dans k noeuds, k étant le degré de réplication du réseau, avec k=|L|+1, L étant la taille des Leaf Set dans les noeuds. Ainsi la donnée V est répliquée sur k noeuds. Néammoins lorsqu'on cherche à lire ou modifier V, on ne va pas s'adresser aux k noeuds, on s'adresse au noeud (qui fait parti de ces k noeuds) dont le NodeID est le plus proche de l'objID de V, on apelle ce noeud N, ce noeud va propager aux k-1 autres noeuds les éventuelles modifications apportées à V.\\

\end{frame}
\begin{frame}{Conciliation des copies}
\framesubtitle{Protocole Quorum-based}
Lors d'une requête vers V (ex:lecture ou modification), on s'adresse donc à N. Et le noeud N va demander leur version de l'objet V aux k-1 autres noeuds qui contiennent l'objet avant d'autoriser la transaction. Si les versions de V envoyés à N sont toutes similaires entre elles alors la transaction s'effectue. Mais si une ou plusieurs versions sont différentes, alors on utilise un protocole Quorum-based.\\C'est à dire que chaque noeud va voter pour sa version de l'objet V et la version qui aura plus de la moitié (le plus souvent) des votes sera la version qui sera stocké par les k noeuds.
\end{frame}
\end{document}
